\documentclass{beamer}

\usepackage{ctex}
\usepackage{graphicx} % Allows including images
\usepackage{booktabs} % Allows the use of \toprule, \midrule and \bottomrule in tables
\usepackage{verbatim}

\mode<presentation>{
% The Beamer class comes with a number of default slide themes
% which change the colors and layouts of slides. Below this is a list
% of all the themes, uncomment each in turn to see what they look like.

%\usetheme{default}
%\usetheme{AnnArbor}
%\usetheme{Antibes}
%\usetheme{Bergen}
%\usetheme{Berkeley}
%\usetheme{Torino}
%\usetheme{Berlin}
%\usetheme{Boadilla}
\usetheme{CambridgeUS}
%\usetheme{Copenhagen}
%\usetheme{Darmstadt}
%\usetheme{Dresden}
%\usetheme{Frankfurt}
%\usetheme{Goettingen}
%\usetheme{Hannover}
%\usetheme{Ilmenau}
%\usetheme{JuanLesPins}
%\usetheme{Luebeck}
%\usetheme{Madrid}
%\usetheme{Malmoe}
%\usetheme{Marburg}
%\usetheme{Montpellier}
%\usetheme{PaloAlto}
%\usetheme{Pittsburgh}
%\usetheme{Rochester}
%\usetheme{Singapore}
%\usetheme{Szeged}
%\usetheme{Warsaw}

% As well as themes, the Beamer class has a number of color themes
% for any slide theme. Uncomment each of these in turn to see how it
% changes the colors of your current slide theme.

%\usecolortheme{albatross}
%\usecolortheme{beaver}
%\usecolortheme{beetle}
%\usecolortheme{crane}
%\usecolortheme{dolphin}
%\usecolortheme{dove}
%\usecolortheme{fly}
%\usecolortheme{lily}
%\usecolortheme{orchid}
%\usecolortheme{rose}
%\usecolortheme{seagull}
%\usecolortheme{seahorse}
%\usecolortheme{whale}
%\usecolortheme{wolverine}

%\setbeamertemplate{footline} % To remove the footer line in all slides uncomment this line
%\setbeamertemplate{footline}[page number] % To replace the footer line in all slides with a simple slide count uncomment this line
%\setbeamertemplate{navigation symbols}{} % To remove the navigation symbols from the bottom of all slides uncomment this line
}

\title[M-V]{一种新的线性非线性反演算法及其在2017年$M_w$7.3伊朗地震滑动分布反演中的应用}
%\author{汇报人:周力璇 \newline 指导老师:许才军 \quad 教授}
\author{汇报人:周力璇}
\institute[武汉大学]
{
武汉大学 \\
\medskip
\textit{schevan@163.com}
}

\date{\today}

\begin{document}

%\titlegraphic{\includegraphics[height=0.1\textwidth]{logo-polito.pdf}{\hspace{250pt}}{\vspace{100pt}}}
\titlegraphic{\includegraphics[height=0.18\textwidth]{whulogo.eps}}
\begin{frame}
\titlepage
\end{frame}

%\logo{\includegraphics[height=0.2\textwidth,angle=0]{whu.eps}{\hspace{-20pt}}{\vspace{-20pt}}}

\begin{frame}
\frametitle{目录}
\tableofcontents
\end{frame}

\section{介绍}
%-----------------------------------------------------
\begin{frame}
\frametitle{问题}

\begin{figure}
  \centering
  % Requires \usepackage{graphicx}
  \includegraphics[scale=0.45]{okada_figure.png}\\
  \caption{Parameters}\label{fig_okada}
\end{figure}

\end{frame}

%-----------------------------------------------------
\begin{frame}
\frametitle{两步法}

\begin{columns}

\column{0.5\textwidth}

\begin{minipage}[c][0.35\textheight][c]{\linewidth}
\begin{block}{第一步:非线性反演}
格网搜索法、拟牛顿法、模拟退火法、遗传算法、邻域法和粒子群算法等
\end{block}
\end{minipage}

\begin{minipage}[c][0.35\textheight][c]{\linewidth}
\begin{block}{第二步:线性反演}
L曲线法、ABIC法、交叉验证法、VCE法和$_j\Re_i$等
\end{block}
\end{minipage}


\end{columns}

\end{frame}
%-----------------------------------------------------
\begin{frame}
\frametitle{贝叶斯反演方法}
\begin{enumerate}
  \item 全贝叶斯反演方法(Fukuda,2008)
  \item 混合线性非线性反演方法(简称F-J方法)(Fukuda,2010)
  \item CATMIP(Minson,2013)
\end{enumerate}
\end{frame}
%-----------------------------------------------------
\begin{frame}
\frametitle{F-J方法和M-V方法}
\begin{table}
  \caption{Parameters solved by Different Methods}
  \label{Tab:Relation_FJ_MV}
  \scalebox{1}{
  \begin{tabular}{llll}
  \hline\noalign{\smallskip}
  Method & Monte Carlo Sampling & Least Squares & VCE method \\
  \noalign{\smallskip}\hline\noalign{\smallskip}
  F-J & $\textbf{m},\boldsymbol\sigma,\alpha^2$ & $\textbf{s}$  & - \\
  \textcolor[rgb]{0.00,1.00,0.00}{M-V} & $\textbf{m}$ & - & $\textbf{s},\boldsymbol\sigma,\alpha^2$ \\
  \noalign{\smallskip}\hline
  \end{tabular}
  }
\end{table}
\end{frame}

\section{模拟算例}
%-----------------------------------------------------
\begin{frame}
\frametitle{反演结果}
\begin{columns}

\column{0.5\textwidth}
\begin{minipage}[c][0.4\textheight][c]{\linewidth}
\begin{figure}
  \centering
  % Requires \usepackage{graphicx}
  \includegraphics[width=0.8\linewidth]{fig_station_syn.png}\\
  \caption{Displacements}\label{fig_station_syn}
\end{figure}
\end{minipage}

\column{0.5\textwidth}
\begin{minipage}[c][0.4\textheight][c]{\linewidth}
\begin{figure}
  \centering
  % Requires \usepackage{graphicx}
  \includegraphics[width=0.8\linewidth]{slip_distribution_syn.png}\\
  \caption{Slip distribution}\label{Fig:slip_distribution_syn}
\end{figure}
\end{minipage}

\end{columns}

\end{frame}

%-----------------------------------------------------
\begin{frame}
\frametitle{反演过程}
\begin{columns}

\column{0.5\textwidth}
\begin{minipage}[c][0.4\textheight][c]{\linewidth}
\begin{figure}
  % Requires \usepackage{graphicx}
  \includegraphics[width=0.6\linewidth]{plot_sample_syn.png}\\
  \caption{Samplings}\label{Fig:sample_syn}
\end{figure}
\end{minipage}

\column{0.5\textwidth}
\begin{minipage}[c][0.4\textheight][c]{\linewidth}
\begin{table}
  \caption{Non-linear Parameters}
  \label{Tab:Syn_Parameters}
  \scalebox{0.45}{
  \begin{tabular}{llllll}
  \hline\noalign{\smallskip}
  Parameter & True Value & Boundary & Initial Value & MaxPV & CI \\
  \noalign{\smallskip}\hline\noalign{\smallskip}
  $m_e$ /km & 0 & [-40,40] & 4 & -0.9 & [-26.91,21.89] \\
  $m_n$ /km & 0 & [-30,30] & 4 & -5.5 & [-27.70,7.72] \\
  $m_u$ /km & 1 & (0,15] & 5 & 3.8 & [0.44,6.07] \\
  $m_s$ /$\circ$& 70 & (0,90) & 60 & 70.5 & [55.27,87.87] \\
  $m_d$ /$\circ$& 15 & (0,90) & 5 & 14.9 & [8.66,18.54] \\
  \noalign{\smallskip}\hline
  \end{tabular}
  }
\end{table}
\end{minipage}

\begin{minipage}[c][0.4\textheight][c]{\linewidth}
\begin{figure}
  % Requires \usepackage{graphicx}
  \includegraphics[width=0.7\linewidth]{plot_SigmaMw_syn.png}\\
  \caption{Relative Weights, $M_w$ and Rake}\label{Fig:plot_SigmaMw_syn}
\end{figure}
\end{minipage}

\end{columns}

\end{frame}

%-----------------------------------------------------
\begin{frame}
\frametitle{反演结果}
\begin{columns}

\column{0.5\textwidth}
\begin{minipage}[c][0.4\textheight][c]{\linewidth}
\begin{figure}
  \centering
  % Requires \usepackage{graphicx}
  \includegraphics[width=0.8\linewidth]{fig_station_syn.png}\\
  \caption{Displacements}\label{fig_station_syn}
\end{figure}
\end{minipage}

\column{0.5\textwidth}
\begin{minipage}[c][0.4\textheight][c]{\linewidth}
\begin{figure}
  \centering
  % Requires \usepackage{graphicx}
  \includegraphics[width=0.8\linewidth]{slip_distribution_syn.png}\\
  \caption{Slip distribution}\label{Fig:slip_distribution_syn}
\end{figure}
\end{minipage}

\end{columns}

\end{frame}

\section{2017年$M_w$7.3伊朗地震}
%-----------------------------------------------------
\begin{frame}
\frametitle{构造背景}
\begin{figure}
  \centering
  % Requires \usepackage{graphicx}
  \includegraphics[width=0.5\linewidth]{fig_tectonic_iran.png}\\
  \caption{Tectonic setting}\label{fig_tectonic_iran}
\end{figure}
\end{frame}
%-----------------------------------------------------
\begin{frame}
\frametitle{反演结果}
\begin{columns}

\column{0.5\textwidth}
\begin{minipage}[c][0.4\textheight][c]{\linewidth}
\begin{figure}
  \centering
  % Requires \usepackage{graphicx}
  \includegraphics[width=0.8\linewidth]{fig_station_Iran.png}\\
  \caption{Displacements}\label{fig_station_Iran}
\end{figure}
\end{minipage}

\column{0.5\textwidth}
\begin{minipage}[c][0.4\textheight][c]{\linewidth}
\begin{figure}
  \centering
  % Requires \usepackage{graphicx}
  \includegraphics[width=0.8\linewidth]{slip_distribution_Iran.png}\\
  \caption{Slip distribution}\label{Fig:slip_distribution_Iran}
\end{figure}
\end{minipage}

\end{columns}

\end{frame}

%-----------------------------------------------------
\begin{frame}
\frametitle{反演过程}
\begin{columns}

\column{0.5\textwidth}
\begin{minipage}[c][0.4\textheight][c]{\linewidth}
\begin{figure}
  % Requires \usepackage{graphicx}
  \includegraphics[width=0.6\linewidth]{plot_sample_Iran.png}\\
  \caption{Samplings}\label{Fig:sample_Iran}
\end{figure}
\end{minipage}

\column{0.5\textwidth}
\begin{minipage}[c][0.4\textheight][c]{\linewidth}
\begin{table}
  \caption{Non-linear Parameters}
  \label{Tab:Iran_Parameters}
  \scalebox{0.5}{
  \begin{tabular}{lllll}
  \hline\noalign{\smallskip}
  Parameter & Boundary & Initial Value & MaxPV & CI \\
  \noalign{\smallskip}\hline\noalign{\smallskip}
  $m_e$ /km &  [480,680] & 550 & 539.42 & [528.20,554.00] \\
  $m_n$ /km &  [3760,3960] & 3800 & 3776.55 & [3770.00,3807.00] \\
  $m_u$ /km &  (0,15] & 1 & 0.20 & [0.01,3.12] \\
  $m_s$ /$\circ$&  (270,360) & 351 & 345.00 & [335.70,358.20] \\
  $m_d$ /$\circ$&  (0,90) & 16 & 13.65 & [10.71,17.61] \\
  \noalign{\smallskip}\hline
  \end{tabular}
  }
\end{table}
\end{minipage}

\begin{minipage}[c][0.4\textheight][c]{\linewidth}
\begin{figure}
  % Requires \usepackage{graphicx}
  \includegraphics[width=0.7\linewidth]{plot_SigmaMw_Iran.png}\\
  \caption{Relative Weights, $M_w$ and Rake}\label{Fig:plot_SigmaMw_Iran}
\end{figure}
\end{minipage}

\end{columns}

\end{frame}

%-----------------------------------------------------
\begin{frame}
\frametitle{比较}
\begin{table}[H]
  \caption{Representative Focal mechanisms for the 2017 $M_w$7.3 Iran Earthquake}
  \label{Tab2:mech_iran}
  \scalebox{0.5}{
  \begin{tabular}{llllllllll}
  \hline\noalign{\smallskip}
    & Longitude & Latitude & Depth & Length & Width & Strike &Dip & Rake & Magnitude   \\
   & ($^\circ$) & (km) & (km) &(km) & ($^\circ$) &($^\circ$) & ($^\circ$) & ($^\circ$) & ($M_w$) \\
  \noalign{\smallskip}\hline\noalign{\smallskip}
  USGS  & 45.96 & 34.91 & 19.0 & 80 & 50 & 351 & 16 & 137 & 7.3  \\
  GCMT$^1$ & 45.84 & 34.83 & 17.9 & - & - & 351 & 11 & 140 & 7.4  \\
  Feng et al.(2018) & 45.87$^a$ & 34.73$^a$ & 14.5$^a$ & 100 & 80 & 353.5 & 14.5 & 135.6 & 7.3  \\
  Barnhart et al.(2018)& 45.87$^a$ & 34.65$^a$ & 15$^a$ & - & - & 351 & 15 & 128 & 7.3  \\
  Huang et al.(2019) &- & - & 14.5$^a$ & 70 & 35 & 355.5 & 17.5 & 135.5 & 7.2 \\
  Nissen et al.(2019) &- & - &- & - & - & 353.7 & 14.3 & 136.8 & 7.25  \\
  Gombert et al.(2019) &- & - & -&84 &56 & 351 & 14 & 131.5 & 7.3  \\
  \textcolor[rgb]{0.00,1.00,0.00}{This study}  & 45.89$^a$ & 34.71$^a$ &13.67$^a$ & 90 & 70 & 345.00$^b$ & 13.65$^b$ & 138.80$^b$ & 7.29$^b$ \\
  \noalign{\smallskip}\hline
  \end{tabular}
  }
  \\
  \footnotesize{$^1$ Global Centroid Moment Tensor}
  \quad \footnotesize{$^a$ The Center of the fault patch which has the largest slip magnitude}
  \quad \footnotesize{$^b$ The values located at the maximum probability}
\end{table}
\end{frame}
\section{讨论和结果}
%-----------------------------------------------------
\begin{frame}
\frametitle{2017年$M_w$7.3伊朗地震反演结果分析}
\begin{columns}

\column{0.5\textwidth}
\begin{minipage}[c][0.8\textheight][c]{\linewidth}
\begin{figure}
  \centering
  % Requires \usepackage{graphicx}
  \includegraphics[width=1\linewidth]{fig_tectonic_iran.png}\\
  \caption{Tectonic setting}\label{fig_tectonic_iran}
\end{figure}
\end{minipage}

\column{0.5\textwidth}
\begin{minipage}[c][0.4\textheight][c]{\linewidth}
\begin{figure}
  \centering
  % Requires \usepackage{graphicx}
  \includegraphics[width=0.8\linewidth]{slip_distribution_Iran.png}\\
  \caption{Slip distribution}\label{Fig:slip_distribution_Iran}
\end{figure}
\end{minipage}

\end{columns}

\end{frame}

%-----------------------------------------------------
\begin{frame}
\frametitle{步长}
\begin{columns}

\column{0.5\textwidth}
\begin{minipage}[c][0.4\textheight][c]{\linewidth}
\begin{figure}
  % Requires \usepackage{graphicx}
  \includegraphics[width=0.6\linewidth]{plot_sample_syn.png}\\
  \caption{Synthetic earthquake}\label{Fig:sample_syn}
\end{figure}
\end{minipage}

\column{0.5\textwidth}
\begin{minipage}[c][0.4\textheight][c]{\linewidth}
\begin{figure}
  % Requires \usepackage{graphicx}
  \includegraphics[width=0.6\linewidth]{plot_sample_Iran.png}\\
  \caption{2017年$M_w$7.3伊朗地震}\label{Fig:plot_sample_Iran}
\end{figure}
\end{minipage}

\end{columns}

\end{frame}

%-----------------------------------------------------
\begin{frame}
\frametitle{先验信息}
\begin{columns}

\column{0.5\textwidth}
\begin{minipage}[c][0.4\textheight][c]{\linewidth}
\begin{table}
  \caption{Synthetic earthquake}
  \label{Tab:Syn_Parameters}
  \scalebox{0.45}{
  \begin{tabular}{llllll}
  \hline\noalign{\smallskip}
  Parameter & True Value & \textcolor[rgb]{1.00,0.00,0.00}{Boundary} & Initial Value & MaxPV & CI \\
  \noalign{\smallskip}\hline\noalign{\smallskip}
  $m_e$ /km & 0 & \textcolor[rgb]{1.00,0.00,0.00}{[-40,40]} & 4 & -0.9 & [-26.91,21.89] \\
  $m_n$ /km & 0 & \textcolor[rgb]{1.00,0.00,0.00}{[-30,30]} & 4 & -5.5 & [-27.70,7.72] \\
  $m_u$ /km & 1 & \textcolor[rgb]{1.00,0.00,0.00}{(0,15]} & 5 & 3.8 & [0.44,6.07] \\
  $m_s$ /$\circ$& 70 & \textcolor[rgb]{1.00,0.00,0.00}{(0,90)} & 60 & 70.5 & [55.27,87.87] \\
  $m_d$ /$\circ$& 15 & \textcolor[rgb]{1.00,0.00,0.00}{(0,90)} & 5 & 14.9 & [8.66,18.54] \\
  \noalign{\smallskip}\hline
  \end{tabular}
  }
\end{table}
\end{minipage}

\column{0.5\textwidth}
\begin{minipage}[c][0.4\textheight][c]{\linewidth}
\begin{table}
  \caption{2017年$M_w$7.3伊朗地震}
  \label{Tab:Iran_Parameters}
  \scalebox{0.45}{
  \begin{tabular}{lllll}
  \hline\noalign{\smallskip}
  Parameter & \textcolor[rgb]{1.00,0.00,0.00}{Boundary} & Initial Value & MaxPV & CI \\
  \noalign{\smallskip}\hline\noalign{\smallskip}
  $m_e$ /km &  \textcolor[rgb]{1.00,0.00,0.00}{[480,680]} & 550 & 539.42 & [528.20,554.00] \\
  $m_n$ /km &  \textcolor[rgb]{1.00,0.00,0.00}{[3760,3960]} & 3800 & 3776.55 & [3770.00,3807.00] \\
  $m_u$ /km &  \textcolor[rgb]{1.00,0.00,0.00}{(0,15]} & 1 & 0.20 & [0.01,3.12] \\
  $m_s$ /$\circ$&  \textcolor[rgb]{1.00,0.00,0.00}{(270,360)} & 351 & 345.00 & [335.70,358.20] \\
  $m_d$ /$\circ$&  \textcolor[rgb]{1.00,0.00,0.00}{(0,90)} & 16 & 13.65 & [10.71,17.61] \\
  \noalign{\smallskip}\hline
  \end{tabular}
  }
\end{table}
\end{minipage}

\end{columns}

\end{frame}
\begin{comment}
%------------------------------------------------
\begin{frame}
\frametitle{Bullet Points}
\begin{itemize}
\item Lorem ipsum dolor sit amet, consectetur adipiscing elit
\item Aliquam blandit faucibus nisi, sit amet dapibus enim tempus eu
\item Nulla commodo, erat quis gravida posuere, elit lacus lobortis est, quis porttitor odio mauris at libero
\item Nam cursus est eget velit posuere pellentesque
\item Vestibulum faucibus velit a augue condimentum quis convallis nulla gravida
\end{itemize}
\end{frame}

%------------------------------------------------
\begin{frame}
\frametitle{Blocks of Highlighted Text}
\begin{block}{Block 1}
Lorem ipsum dolor sit amet, consectetur adipiscing elit. Integer lectus nisl, ultricies in feugiat rutrum, porttitor sit amet augue. Aliquam ut tortor mauris. Sed volutpat ante purus, quis accumsan dolor.
\end{block}

\begin{block}{Block 2}
Pellentesque sed tellus purus. Class aptent taciti sociosqu ad litora torquent per conubia nostra, per inceptos himenaeos. Vestibulum quis magna at risus dictum tempor eu vitae velit.
\end{block}

\begin{block}{Block 3}
Suspendisse tincidunt sagittis gravida. Curabitur condimentum, enim sed venenatis rutrum, ipsum neque consectetur orci, sed blandit justo nisi ac lacus.
\end{block}
\end{frame}

%------------------------------------------------
\begin{frame}
\frametitle{Multiple Columns}
\begin{columns}[c] % The "c" option specifies centered vertical alignment while the "t" option is used for top vertical alignment

\column{.45\textwidth} % Left column and width
\textbf{Heading}
\begin{enumerate}
\item Statement
\item Explanation
\item Example
\end{enumerate}

\column{.5\textwidth} % Right column and width
Lorem ipsum dolor sit amet, consectetur adipiscing elit. Integer lectus nisl, ultricies in feugiat rutrum, porttitor sit amet augue. Aliquam ut tortor mauris. Sed volutpat ante purus, quis accumsan dolor.

\end{columns}
\end{frame}

%------------------------------------------------
\begin{frame}
\frametitle{Table}
\begin{table}
\begin{tabular}{l l l}
\toprule
\textbf{Treatments} & \textbf{Response 1} & \textbf{Response 2}\\
\midrule
Treatment 1 & 0.0003262 & 0.562 \\
Treatment 2 & 0.0015681 & 0.910 \\
Treatment 3 & 0.0009271 & 0.296 \\
\bottomrule
\end{tabular}
\caption{Table caption}
\end{table}
\end{frame}

%------------------------------------------------
\begin{frame}
\frametitle{Theorem}
\begin{theorem}[Mass--energy equivalence]
$E = mc^2$
\end{theorem}
\end{frame}

%------------------------------------------------
\begin{frame}[fragile] % Need to use the fragile option when verbatim is used in the slide
\frametitle{Verbatim}
\begin{example}[Theorem Slide Code]
\begin{verbatim}
\begin{frame}
\frametitle{Theorem}
\begin{theorem}[Mass--energy equivalence]
$E = mc^2$
\end{theorem}
\end{frame}\end{verbatim}
\end{example}
\end{frame}

%------------------------------------------------
\begin{frame}
\frametitle{Figure}
Uncomment the code on this slide to include your own image from the same directory as the template .TeX file.
%\begin{figure}
%\includegraphics[width=0.8\linewidth]{test}
%\end{figure}
\end{frame}

%------------------------------------------------

\begin{frame}[fragile] % Need to use the fragile option when verbatim is used in the slide
\frametitle{Citation}
An example of the \verb|\cite| command to cite within the presentation:\\~

This statement requires citation \cite{p1}.
\end{frame}

%------------------------------------------------

\begin{frame}
\frametitle{References}
\footnotesize{
\begin{thebibliography}{99} % Beamer does not support BibTeX so references must be inserted manually as below
\bibitem[Smith, 2012]{p1} John Smith (2012)
\newblock Title of the publication
\newblock \emph{Journal Name} 12(3), 45 -- 678.
\end{thebibliography}
}
\end{frame}
\end{comment}

\section{}
\begin{frame}
\Huge{\centerline{谢谢}}
\end{frame}

\end{document}
